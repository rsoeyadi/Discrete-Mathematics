\documentclass{article}
\usepackage{amssymb}
\usepackage{amsmath}



\title{Homework 6 - Relations (Spring 2021)}
\author
{
Ryan, Soeyadi
\and rs4163
}


\begin{document}
    \maketitle
    
    \section*{Question 1}
    4, 9, 5, 8, 7, 1, 3, 6, 2
    \newpage
    \section*{Question 2}
    Set $A = \{10,20,30,40,50\}$ \\Let $R$ be a relation defined on $A$ \\  
    
    1. $R$ is reflexive would require that the pairs \\ \\
    \indent \indent$\{(10,10), (20,20), (30,30), (40,40), (50,50)\}$ must be in $R$ \\ \\
     \indent 2. $R$ is symmetric would require no pairs, based on the fact that the \\ \indent empty set is vacuously symmetric.  \\
  
    3. If $R$ is antisymmetric, and $(x,y)$ and $(y,x)$ belong to $R$\\
    \indent if $R$ is in fact antisymmetric \\
    \indent $x = y$, since antisymmetry exists even if we have a pair such as $(3,3)$ or \indent$(5,5)\ \in R$\\ \\
    4. Let $R = \{(10,20)\}$ \\
    \indent $R$ is antisymmetric and transitive
    
    \newpage
    \section*{Question 3}
1. $S$ as a Boolean matrix \\
\begin{center}
\begin{tabular}{ |c|c|c|c|c| } 
 \hline
\ & 0 &1 & 2 & 3\\ \hline
0 & 1 &1 & 0 & 1\\ \hline
1 & 1 &1 & 0 & 0\\ \hline
2 & 0 & 0 & 1 & 0 \\  \hline
3 & 1 &0 & 0 & 1\\ \hline
\end{tabular}
\end{center} 
2. $S^{-1}$ =  \{(0,0), (0,1), (0,3), (1,0), (1,1), (2,2), (3,0), (3,3)\} \\ \\ 
3. $S$ =  \{(0,0), (0,1), (0,3), (1,0), (1,1), (2,2), (3,0), (3,3)\} \\ \\
\indent $S$ is symmetrical and reflexive \\
\indent $S^{-1}$ is symmetrical and reflexive \\
\indent $R$ is irreflexive and antisymmetric and transitive \\
\indent $T$ is non-reflexive and antisymmetric \\
    \newpage
    \section*{Question 4}
  1. 
   Let $x,y \in \mathbb{R} $ \\ 
   \indent
  $xRy \ IFF \ (x - y) \in \mathbb{Z}$ \\ \\
   \indent
  Reflexivity: $\forall x \in A \ x \ R \ x$ \\
   \indent
  \indent $x - x = 0$ \\ 
   \indent
  \indent $0 = 0$ \\
   \indent
  \indent $0 \ \in \mathbb{Z}$ \\ \\
   \indent
  Symmetric: $x - y \in \mathbb{Z} \implies y - x \in \mathbb{Z}$ \\
   \indent
  $x - y \implies (-x + y)$ \\
   \indent
  $x - y \implies -(x - y)$ \\
   \indent
  Since $x - y \in \mathbb{Z}$, the negation of it is also an integer \\ \\
   \indent
  Transitive: $\forall x,y,z \in A (\ x R y \land y R z \implies \ x R z)$ \\
   \indent
  $x - y + y - z$ \\
   \indent
  $x - z \in \mathbb{Z}$ \\ \\
  2. $[1.75]$ \\
  $\{...-3.25, -2.25, -1.25...\} \cup \{...0.75, 1.75, 2.75...\}$
    \newpage
    \section*{Question 5}
    1. 
    
    
    \indent $[1] =  \{\pm 1, \pm 3, \pm 5...\}$ \\
    \indent $[2] = \{\pm 2, \pm 4, \pm 6...\}$ \\
    \indent $[3] =  [1]$ \\
    \indent $[4] = [2]$ \\
    \indent $[5] = [1]$ \\
    \indent $[6] = [2]$ \\ \\
    2.
    	There are 2 distinct equivalence classes; once we go past [2] and so on, we repeat the evens and the odds.
    \newpage
    \section*{Question 6}
    1. \\
     \indent $[0] =  \{0, -4, 4, -8, 8, -12, 12...\}$ \\
       \indent $[1] =  \{1, 1 \pm 4, 1 \pm 8, 1 \pm 12...\}$ \\
     \indent $[1] =  \{1, 1 \pm 4, 1 \pm 8, 1 \pm 12...\}$ \\
    \indent $[2] =  \{2, 2 \pm 4, 2 \pm 8, 2 \pm 12...\}$ \\   
    \indent $[3] =  \{3, 3 \pm 4, 3 \pm 8, 3 \pm 12...\}$ \\
    \indent $[4] =  [0]$ \\
    \indent $[5] = [1]$ \\    \\
    2. There are four distinct equivalence classes; once we reach $[4]$, we start repeating the sets from $[0]$ and so on; the remainders are found from the same integers. \\  \\
    3. The equivalence classes are disjoint since they are not equal, and they make up all of the integers.
   \end{document}