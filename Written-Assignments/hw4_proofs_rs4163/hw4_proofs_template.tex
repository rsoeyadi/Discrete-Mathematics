\documentclass{article}
\usepackage{amsfonts} 
\usepackage{amsthm}

\title{Homework 4 - Proofs (Spring 2021)}
\author
{
First Name, Last Name
\and UNI
}


\begin{document}
    \maketitle
    
    \section*{Question 1}
    \textbf{Direct Proof} to prove: \\
    \indent For all integers $n$, $3$ divides $(3n + 1)(3n + 2)(3n + 3)$. \\ \\ 
    Let $n \in \mathbb{Z}$ \\
    We are trying to prove $3 | (3n + 1)(3n + 2)(3n + 3)$ \\ \\
    Let $a \in \mathbb{Z}$ \\
    $a = (3n + 1)(3n + 2)(3n + 3)$ \\
    $\equiv (27n^3 + 54n^2 + 33n + 6)$ \\
    $\equiv 3(9n^3 + 18n^2 + 11n + 2)$ \\ \\
    $\frac{a}{3} = (9n^3 + 18n^2 + 11n + 2)$ \\ \\
    Since $\frac{a}{3}$ is equal to $(9n^3 + 18n^2 + 11n + 2)$, we have proved that \\ for all integers $n$, $3$ divides $(3n + 1)(3n + 2)(3n + 3)$. \\ \qed

    \newpage
    \section*{Question 2}
   \textbf{Proof by Contrapositive} to prove: \\
   \indent For any integer $n$, if $3n + 1$ is even, then $n$ is odd. \\ \\
   The \textbf{contrapositive} is: \\
   \indent If $n$ is even, then $3n + 1$ is odd \\ \\
   Let $n \in \mathbb{Z}$ \\
   Let $a \in \mathbb{Z}$ \\ 
   Let $b \in \mathbb{Z}$ \\ \\
   $n = 2a$, since we assume $n$ is even \\ \\
   $b = 3n + 1$, if in fact $3n + 1$ is odd \\
   $\equiv 3(2a) + 1$ \\
   $\equiv 6a + 1$ \\
   $\equiv 2(3a) + 1$ \\ \\
   Let $c \in \mathbb{Z}$ \\ \\
   $c = 3a$ \\
   $b = 2(c) + 1$ \\
   $\equiv 2c + 1$, the form of an odd integer \\ \\
   Since $b$ is now in the form of an odd integer, we have proven the \\ contrapositive. We can conclude that for any integer $n$, if $3n + 1$ is even, then $n$ is odd. \\ \qed
   
   
    \indent
    \newpage
    \section*{Question 3}
    \textbf{Proof of IFF} to prove: \\
    \indent Let $x, y$ be integers, and prove that the product $xy$ is odd if and only if $x$ and $y$ are both odd integers. \\ \\
    We must prove both: \\
    \indent if $xy$ is odd, then $x$ and $y$ are both odd integers \\
    \indent if $x$ and $y$ are both odd integers, then $xy$ is odd \\ \\
    For the first statement, we may prove the \textbf{contrapositive} of: \\
    \indent if $x$ and $y$ are both \textbf{even} integers, then $xy$ is even \\ \\
    Let $n \in \mathbb{Z}$ \\
    $x = 2n$, since we assume $x$ is even \\
    $y = 2n$, since we assume $y$ is even \\ \\
    $xy = (2n)(2n)$ \\
    $\equiv 4n^2$ \\
    $\equiv 2(2n^2)$ \\ \\
    Let $a \in \mathbb{Z}$ \\
    $a = 2n^2$ \\
    $xy = 2a$ \\ \\
    $xy$ is now in the form of an even integer. \\
    We have proven that if $x$ and $y$ are both \textbf{even} integers, then $xy$ is even. \\ \\
     For the second statement, we may prove: \\
    \indent if $x$ and $y$ are both \textbf{odd} integers, then $xy$ is odd \\ \\
    Let $n \in \mathbb{Z}$ \\
    $x = 2n + 1$, since we assume $x$ is odd \\
    $y = 2n + 1$, since we assume $y$ is odd \\ \\
    $xy = (2n + 1)(2n + 1)$ \\ 
    $\equiv 4n^2 + 4n + 1$ \\
    $\equiv 2(2n^2 + 2n) + 1$\\ \\
    Let $k \in \mathbb{Z}$ \\
    $k = 2n^2 + 2n$ \\
    $xy = 2k + 1$ \\ \\
    $xy$ is now in the form of an odd integer. \\
    We have proven that if $x$ and $y$ are both \textbf{odd} integers, then $xy$ is odd.  \\ \qed
    
    \newpage
    \section*{Question 4}
    \textbf{Proof by cases} of: \\
    \indent if $n$ is an integer, then $n^2 \geq n$ \\ \\
    Let $n \in \mathbb{Z}$ \\ \\
    We have three possible cases since $n \in \mathbb{Z}$: \\
    \indent $n < 0$, $n = 0$, and $n > 0$ \\ \\
    \textbf{Case 1}, $n < 0$ \\ \\
    \indent Let $a \in \mathbb{Z}$ \\
    \indent $(-a)^2 \geq (-a)$ \\
    \indent $a^2 \geq -a$ \\ 
    \indent $n = -a$ \\
    \indent $n^2 \geq n$ \\ \\ \\
    \textbf{Case 2}, $n = 0$ \\ \\
    \indent Let $b \in \mathbb{Z}$ \\
    \indent $(0)^2 \geq (0)$ \\
    \indent $0 \geq 0$ \\ 
    \indent $n = 0$ \\
    \indent $n^2 \geq n$ \\ \\ \\
    \textbf{Case 3}, $n > 0$ \\ \\
    \indent Let $c \in \mathbb{Z}$ \\
    \indent $(a)^2 \geq (a)$ \\
    \indent $a^2 \geq a$ \\ 
    \indent $n = a$ \\
    \indent $n^2 \geq n$  \\ \qed
    
    \newpage
    \section*{Question 5}
    \textbf{Proof by Counter Example} of: \\
    \indent For every prime number $p > 2$, there exists a natural number $n$\\ \indent such that $p = 2^n - 1$ \\ \\
    Consider the prime number $5$: \\ \\
    \indent There is no $n \in \mathbb{N}$ such that $5 = 2^n -1$ \\ \\
    \indent $5 = 2^n - 1$ \\
    \indent $6 = 2^n$ \\ \\
    There does not exist $n \in \mathbb{N}$ such that we can make this statement True. \\ \qed
    \newpage
    \section*{Question 6}
    \textbf{Proof by contradiction} of: \\
    \indent For all integers $n \in \mathbb{Z}$, if $n^2$ is odd, then $n$ is odd. \\ \\
    We can assume the negation, if this statement is False:\\
    \indent $n^2$ is odd and $n$ is even \\ \\
    Let $n \in \mathbb{Z}$ \\
    Let $a \in \mathbb{Z}$ \\ \\
    $n = 2a$, since $n$ is even \\
    $n^2 = (2a)^2$ \\
    $\equiv 4a^2$ \\ \\
    We have run into a contradiction since $n$ is even \textbf{and} $n^2$ is even, meaning that the opposite, our initial statement, is True. \\ \qed
    
  

   \end{document}